% General packages

\usepackage{amsmath}
\usepackage{amsthm}
\usepackage{amssymb}
\usepackage{hyperref}
\usepackage{cleveref}

% Environments

\newtheorem{theorem}{Theorem}
\newtheorem{lemma}[theorem]{Lemma}
\newtheorem{proposition}[theorem]{Proposition}

\theoremstyle{definition}
\newtheorem{definition}[theorem]{Definition}

\theoremstyle{remark}
\newtheorem{remark}[theorem]{Remark}
\newtheorem{comment}[theorem]{Comment}
\newtheorem{Notations}[theorem]{Notations}


% Mathematical Notations: Fields

\newcommand{\naturalN}{\mathbb{N}}
\newcommand{\integerN}{\mathbb{Z}}
\newcommand{\rationalN}{\mathbb{Q}}
\newcommand{\realN}{\mathbb{R}}
\newcommand{\complexN}{\mathbb{C}}

% Mathematical Notations: Basics

\newcommand{\abs}[1]{\mathopen{|}#1\mathclose{|}} % Absolute value / modulus
\newcommand{\size}[1]{\mathrm{card}(#1)} % cardinal / size of set / graph / ... 
\newcommand{\identity}[1][]{\mathrm{Id}_{#1}} % Identity 

% Mathematical Notations: Matrices

\newcommand{\diag}[1]{\mathrm{diag}(#1)}   % Diagonal matrices defined from the list of coeficients
\newcommand{\matrices}[2][n]{\mathcal{M}_{#1}(#2)}   % Algebra of #1 matrices -- for square matrices, we write #1=n instead of $n\times n$.

\newcommand{\ST}[2][n]{\mathbf{ST}_{#1}(#2)}

\newcommand{\supp}[1]{\mathrm{supp}(#1)} % Support: used in lecture-11 with argument a matrix, returning the 0-1-matrix which is the adjacency matrix of the underlying graph.

% Mathematical Notations: Groups and Actions

\newcommand{\permutationgroup}[1][n]{\mathfrak{S}_{#1}} % group of permutations on n elements sets.

\newcommand{\nullcone}[1]{\mathcal{N}(#1)}



